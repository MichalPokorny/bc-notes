\begin{itemize}
\item Linearni mnoziny ve vektorovem prostoru, jejich geometricka interpretace.
\item Reseni soustavy rovnic je linearni mnozina.
\item Frobeniova veta.
\item Reseni soustavy upravou matice.
\item Souvislost soustavy reseni s ortogonalnim doplnkem.
\end{itemize}

Afinni prostor. Linearni zobrazeni: $f^{-1}(b)=\{x_0+Ker f\}$ kde $x_0$
je libovolny vektor z $f^{-1}(b)$ (protoze $Ker f$ je podprostor a $f$ je
linearni).

Afinni kombinace: $\sum\alpha_i x_i: \sum\alpha_i=1$.

Soustava linearnich rovnic: $Ax=b$.
Mnozina reseni je bud prazdna, nebo $\{x_0+L\}$, kde $x_0$ je nejake reseni
$Ax=b$ a $L$ je mnozina reseni $Ax=0$.
Kazdy afinni podprostor jde popsat soustavou linearnich rovnic.
Prunik af. prostoru je afinni podprostor.

Frobeniova veta: $rank(A)=rank((A b)) \leftrightarrow Ax=b$ ma aspon jedno
reseni. Dukaz: je resenim $\leftrightarrow A_1\alpha_1+\ldots A_n\alpha_2=b$
prave kdyz $b$ je LK sloupcu $A_i\rightarrow b\in span(A_1\ldots A_n)$,
takze $span(b,...)=span(...)$. $rank(A)=dim(sloupce)=dim(sloupce,b)=rank(A,b)$.

Reseni upravou: elementarni upravy: $\cdot\alpha\neq 0$, vynasob a pricti jinam,
vymena.

REF (odstupnovany tvar), RREF, Gaussova eliminace.

Souvislost s ortogonalnim doplnkem: $\{\langle x,v\rangle=0 \forall v\}$
Mnozina reseni homogenni soustavy je ortogonalni doplnek radku matice.
