\begin{itemize}
\item Definice, základní vlastnosti determinantu
\item Úpravy determinantu, výpočet.
\item Geometrický smysl determinantu.
\item Minory a inverzní matice.
\item Cramerovo pravidlo.
\end{itemize}

$\det A=\sum_{\sigma} sgn(\sigma)\prod a_{i,\sigma(i)}$

$\det A=\det A^T$ protoze $sgn(p)=sgn(p^{-1})$.

Prerovnani radku muze otocit znamenko je-li to licha permutace.

Determinant je lin. funkce kazdeho radku a kazdeho sloupce:
$\det((stuff) radek a (stuff))+\det((stuff) radek b (stuff))=\det((stuff) radek
a+b (stuff))$

$\det(AB)=\det(A)\det(B)$
Vypocet pres Gaussovu eliminaci.
Geometricky vyznam: pravotocivy/levotocivy objem.

Determinant $A_{ij}$ := minor $a_{i,j}$ matice $A$
$\det(A)=\sum a_{i,j}(-1)^{i+j}\det(A_{ij})$

Adjungovana matice: $(adj A)_{i,j}=(-1)^{i+j}\det(A_{ji})$
$A^{-1}=(adj A)/\det(A)$, protoze to da rozvoj podle radku nebo matici
s opakovanim radku.

Cramerovo pravidlo: $x=det(A_i)/det(A)$, kde $A_i$ dostanu nahrazenim $i$-teho
sloupce $b$ckem. Dukaz pres $adj$.
