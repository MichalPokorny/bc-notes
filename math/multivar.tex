\begin{itemize}
\item Parcialni derivace a totalni diferencial.
\item Vety o stredni hodnote.
\item Extremy funkci vice promennych.
\item Veta o implicitnich funkcich.
\end{itemize}

Parcialni derivace, derivace ve smeru, gradient.
Totalni diferencial: existuje-li $\lim_{\|h\|\imp 0}
\frac{f(X+h)-f(X)-Df(X)(h)}{\|h\|}=0$

Druhy diferencial: $D^2 f(a)(h,k)=\sum\sum\frac{\partial^2 f}{\partial
x_i\partial x_j}(a)h_i h_k$

Bilinearni formy: pozitivne/negativne definitni, indefinitni.
Sylvestrovo kriterium: vsechny hlavni subdeterminanty kladne $\imp F$ pozitivne
definitni. (Alternujici znamenka: negativne definitni.)

Ma-li $f$ totalni diferencial, tak $D_v f(a)$ jsou pro $v\neq o$ vlastni a $D_v
f(a)=Df(a)(v)$ a $Df(a)(h)=\langle\nabla f(a),h\rangle$ a $\partial f$ vsechny
existuji a $f$ je spojita v $a$. Totalni diferencial ma aritmetiku jako
derivace.

Retizkove pravidlo (diferencial slozeneho zobrazeni): $\frac{\partial
H}{\partial x_i}(a)=\sum_{j=1}^n \frac{\partial f}{\partial y_i}(b)
\frac{\partial g_j}{\partial x_i}(a)$

Totalni diferencial existuje kdyz jsou spojite vsechny $\partial$.
Druhy diferencial existuje kdyz jsou spojite vsechny $\partial^2$.
Kdyz je $\partial^2 f/\partial x\partial y$ spojita, tak se da zamenit poradi.

Veta o stredni hodnote:
$f$ ma vsechny $\partial$ spojite na $(a,b)\imp\e\xi\in(0,1):$
$$f(b)-f(a)=\nabla f(a+\xi(b-a))\cdot (b-a)=\sum_{i=1}^n \frac{\partial
f}{\partial x_i}(a+\xi(b-a))(b_i-a_i)$$
Dukaz z Lagr. vety pro $F(t)=f(a+t(b-a))$ a retizku.

Veta o implicitni funkci pro $\R^2$:
$F([x,y]):\R^2\imp\R$ at ma spojite $\partial$. At $x_0,y_0\in\R$ s.t.
$F([x_0,y_0])=0$, $\frac{\partial F}{\partial y}([x_0,y_0])\neq 0$.
Pak $\e U:P(x_0),V:P(y_0):\a x\in U \existsone y=\varphi(x)\in V: F([x,y])=0$.
Navíc $\frac{\partial\varphi}{\partial x}(x)=-\frac{\frac{\partial F}{\partial
x}}{\frac{\partial F}{\partial y}}([x,\varphi(x)])$

Pro $\R^{n+1}$:
$F:G\imp\R$, $G\subseteq\R^{n+1}$ otevrena. $x_0\in\R^n,y_0\in\R$ s.t.
$[x_0,y_0]\in G, F([x_0,y_0])=0$. At $F$ ma spoj. $\partial$ a $\frac{\partial
F}{\partial y}([x_0,y_0])\neq 0$. Pak $\e U:P(x_0),V:P(y_0):\forall U\existsone
y=\varphi(x)\in V:F([x,y])=0$, $\varphi$ ma spojite $\partial$ a
$\frac{\partial\varphi}{\partial x_i}(x)=-\frac{\frac{\partial F}{\partial
x_i}}{\frac{\partial F}{\partial y}}([x,\varphi(x)])$

TODO: jsou nejake jine vety o stredni hodnote?

TODO TODO TODO

Stacionarni bod: vsechny $\partial$ existuji a jsou 0.
Pokud v lok. extr. existuji vsechny $\partial$, jsou 0.

At ma $F$ spojite $\partial$ druheho radu. Kdyz $Df(a)=0$,
tak:
$D^2 f(a)$ pozitivne definitni $\imp a$ lokalni minimum, negativne $\imp$
maximum, indefinitni $\imp$ neni lokalni extrem.

TODO: Veta o vazanych extremech (Lagrangeovy multiplikatory):
At $G\subseteq\R^n$ je otevrena. $F,g_1,\ldots g_m$, $m<n$, at maji vsechny
spojite $\partial$. $M:=\{g_1(x)=\ldots g_m(x)=0\}$.
Kdyz $a=[a_1,\ldots a_n]$ je lok. extrem $F$ na $M$ a $\nabla g_1(a),\ldots
\nabla g_m(a)$ jsou linearne nezavisle, tak $\e\lambda_1\ldots\lambda_m$:
$DF(a)+\lambda_1 D_{g_1}(a)+\ldots\lambda_m D_{g_m}(a)=0$.
(Neboli: $\frac{\partial F}{\partial x_i}(a)=\sum_{k=1}^m \lambda_k
\frac{\partial g_k}{\partial x_i}(a)$)
