\begin{itemize}
\item Grupa, okruh, teleso - definice, priklady
\item Malá Fermatova věta
\item Dělitelnost a ireducibilní rozklady polynomů
\item Rozklady polynomů na kořenové činitele pro $\R$, $\Q$, $\C$ koeficienty
\item Násobnost kořenů a jejich souvislost s derivacemi mnohočlenu
\end{itemize}

Algebra: zobrazení s aritou. 1 binární: grupoid. Asociativní s neutr. prvkem:
monoid. Existuji-li inverzni prvky, jsou stejne zleva a zprava.
Grupa, abelovská grupa.

Okruh: $(R,+,-,0)$ komutativní, $(R,\cdot,1)$ monoid, $a(b+c)=ab+ac$,
$(a+b)c=ac+bc$. Napriklad $\Z_N$, lin. zobrazeni.

$0a = a0 = 0$. $(-a)b=a(-b)=-(ab)$. $(-a)(-b)=ab$. $|R|>0\leftrightarrow 0\neq
1$.

Těleso: okruh s dělením, $0\neq 1$. Komutativní: má komutativní $\cdot$.

Veta (Wedderburnova): vsechna konecna telesa jsou komutativni.

Mala Fermatova veta: kdyz $a\perp n$, tak $a^\varphi(n)\equiv 1\pmod n$.
Dokaze se z $a^|\Z_n^*|\pmod n=1$, coz plati protoze to je cyklicka grupa,
rad podgrupy deli rad grupy.

Delitelnost a ireducibilni rozklady polynomu.
KMK: $\a a,b,c: ac=bc\imp a=b$. $a|b:=\e c: b=ac$. $a||b:=a|b\wedge b|a$.
OI: komutativni okruh, kde $ab=0\rightarrow a=0\vee b=0$.
Pro kazdy OI je $(R\smallsetminus\{0\},\cdot,1)$ KMK.

(Faktoralgebra dle $||$ uvnitr KMK.)

NSD: $\a i: c|a_i$ a kdyz $\a i: d|a_i$, tak $d|c$.

Ireducibilni prvek: neinvertibilni a $c=ab\rightarrow c\|a\vee c\|b$.
Prvočinitel: není invertibilní a $c|ab\rightarrow c|a\vee c|b$.

V KMK je kazdy prvocinitel ireducibilni. Kdyz maji kazde 2 prvky NSD,
pak je kazdy ired. prvek prvocinitel.
% Vynechano: Vlastnosti NSD (d=NSD(a,b)&e=NSD(ac,bc) ==> dc||e,
%	1=NSD(a,b)&a|bc&NSD(ac,bc) existuje ==> a|c

Polynomy: definovane nad okruhem a monoidem ($R[M]$). $R[\N_0]$: polynomy.
$R[x]$ je OI $\leftrightarrow x$ je OI.
$deg([])=-1$. $deg(pq)\leq deg(p)+deg(q)$

Eukleidovsky obor integrity: Eukleidovska funkce: $a|b,b\neq 0\imp v(a)\leq
v(b)$, da se najit zbytek po deleni z mensi E. fci. Eukleiduv algoritmus.

Rozklady na korenove cinitele.
Polynom je ireducibilni, kdyz neni soucin polynomu nizsich stupnu, stupen $\geq
1$. Az na asociovanost je jednoznacny rozklad.
$j_\alpha:R[x]\rightarrow S$ pro $\alpha\in S$ dosazovaci homomorfismus.
Koren: $j_\alpha=0$
Na OI je $\alpha$ koren $\leftrigharrow (x-\alpha)|p$.

Algebraicky uzavrene teleso: kazdy polynom ma koren.
$\C$ je priklad. $\R$ koeficienty $\imp$ koreny jsou komplexne sdruzene.
Kazdy polynom v $\R[x]$ licheho stupne ma aspon jeden realny koren.

V $\Q$ $\e$ ireducibilni polynomy libovolneho stupne $\geq 1$.
TODO: nemam dukaz.

V $\R$ $\e$ ired. polynomy stupne 2.

Kazdy nenul. polynom ma $\leq\deg p$ korenu. Vicenasobny koren
$\leftrightarrow$ koren derivace.

Charakteristika OI: rad grupy generovane jednickou v +.
Derivace snizuje stupen (n-1).
$NSD(p,p')=1\leftrightarrow$ nema nasobne koreny.
V OI $x^n-1$, $x^{n+1}-x$ nemaji vicenasobny koren (kdyz char nedeli $n$).
