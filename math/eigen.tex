\begin{itemize}
\item Vlastní čísla a vlastní hodnoty lineárního operátoru, resp. čtvercové
matice.
\item Výpočet, základní vlastnosti.
\item Uvedení matice na diagonální tvar v případě různých vlastních čísel.
\item Informace o Jordanově tvaru v obecném případě.
\end{itemize}

$Ax=\lambda x$, $|A_\lambda I|=0$.
Podobnost: $B=P^{-1} AP$.
Diagonalizovatelnost: podobna diagonalni matici.
Diagonalizovatelny linearni operator.

$\chi_A$ je polynom stupne $n$, vedouci koeficient $=(-1)^n$.
Koreny jsou prave $\lambda$'s.
$A$ ma $n$ vlastnich cisel, kdyz je pocitame (nasobnost korene)-krat.
Kdyz je $A$ realna, tak ma komplexne sdruzena vlastni cisla.

Determinant = součin vlastních čísel.
Trace = součet vlastních čísel.
Vlastní čísla reálné symetrické matice jsou reálná.
Podobne matice maji stejny charakteristicky polynom:
$\det(P^{-1}AP-tI)=\det(P^{-1})\det(A-tI)\det(P)$.

$AB$ ctvercove stejneho typu $\imp AB$, $BA$ maji stejna vlastni cisla.

Matice je diagonalizovatelna prave kdyz $\e$ baze z jejich vlastnich vektoru.
$AR=RD$, sloupce $R$ jsou vlastni vektory prislusne vlasnim cislum.
Kdyz jsem diagonalizovatelna, tak na te diagonale mam vsechna vlastni cisla.

Vlastni vektory ruznych vlastnich cisel jsou linearne nezavisle.
$n$ ruznych vlastnich cisel $\rightarrow$ diagonalizovatelna.

Diagonalizovatelnost $\leftrightarrow$
algebraicka nasobnost vsech $\lambda$ je $\dim Ker(A-\lambda I)$.

Spektralni veta: $A$ diagonalizovatelna $\leftrightarrow \e E_1\ldots E_t$
řádu $n$, kde $A=\lambda_1 E_1+\ldots \lambda_t E_t$,
$\a i: E_i^2=E_i$, $E_i E_j=0$, $\sum E_i=I_n$.

Dale pro $A$ plati: $E_t$ jsou jednoznacne urcene matici $A$ a vlastnostmi,
hodnost kazde $E_i$ je alg. nasobnost $\lambda_i$,
pro libovolny polynom $f(x)=c_0+c_1 x+\ldots c_k x^k$ s kompl. koef. plati
$f(A)=c_0 I_n + c_1 A+\ldots c_k A^k$.
$AB=BA$ prave kdyz komutuje s kazdou matici $E_t$.

(Ale tohle se snad hodit moc nebude.)

Jordanova bunka: lambdy na diagonale, 1 hned nad diagonalou.
Velikost Jordanovy bunky je $rank(A-\lambda I)$.

Hermitovska transpozice: $A^H=\bar{A^T}$. Unitarita: $A=A^H$.
