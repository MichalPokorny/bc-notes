\begin{itemize}
\item Uspořádané množiny
\item Množinové systémy, párování, párování v bipartitních grafech (systémy
různých reprezentantů)
\item Kombinatorické počítání
\item Princip inkluze a exkluze
\end{itemize}

Relace: ireflexivni: $xRy\imp x\neq y$. Antisymetricka: $xRy\wedge yRx\imp x=y$
Ekvivalence: symetricka, reflexivni, tranzitivni
Usporadani: reflexivni, tranzitivni a antisymetricka
Linearni usporadani; bezprostredni predchudce
"Po slozkach": $(a,b)\leq(c,d):=a\leq_1 c \wedge b\leq_2 d$,
lexikograficky.

Predusporadani: reflexivni a tranzitivni (nemusi byt antisymetricka)

Spernerova věta: nezávislý systém podmnožin má nejvýš $\binom{n}{\lfloor
n/2\rfloor}$ množin

TODO: dobre usporadani? Zermelova veta/princip dobreho usporadani?

Parovani: podmnozina hran, vrchol maximalne u jedne.
Hallova veta: existuje SRR $\leftrightarrow \forall J\subseteq I:
|\bigcup_{j\in J}M_j|\geq |J|$
Perfektni parovani. Maximalni parovani.

Specialne pro bipartitni graf: $\deg u\geq\deg v\forall u\in A,v\in
B\rightarrow$ existuje parovani velikosti $A$. Kdyz mam v bip. grafu
vsechny vrcholy stejneho stupne, ma perfektni parovani.

Tutteho veta: $G$ ma PP $\leftrightarrow \forall A\subseteq V:$
$c_l(G\smallsetminus A)\leq |A|$ (pocet lichych komponent je mensi nez velikost
mnoziny).

Edmondsuv algoritmus: kontrakce kyticky, nebo stridava cesta.

Existuje $2^{n-1}$ lichych/sudych podmnozin.
Prostych zobrazeni z $[n]$ do $[m]$: $m(m-1)\ldots (m-n+1)$

$\binom{n}{k}=\binom{n}{n-k}$. $\binom{n-1}{k-1}+\binom{n-1}{k}=\binom{n}{k}$.

Nezaporne cislo $m$ jde zapsat jako soucet $r$ scitancu pres
$\binom{m+r-1}{r-1}$ zpusobu.

Binomicka veta: $(x+y)^n=\sum_{k=0}^n\binom{n}{k}x^k y^{n-k}$
(Multinomicka veta pro vice scitancu.)

Princip inkluze a exkluze:
$\left|\bigcup_{i=1}^m A_i\right|=\sum_{k=1}^n (-1)^{k-1}
\sum_{I\in\binom{[n]}{k}}|\bigcap_{i\in I}A_i|$
TODO: dukaz

% Kaslu na projektivni roviny a latinske ctverce
