\begin{itemize}
\item Vlastnosti v $\R$, $\C$ případě
\item Norma
\item Cauchy-Schwarzova nerovnost
\item Kolmost
\item Ortogonální doplněk a jeho vlastnosti
\end{itemize}

Vlastnosti: $\langle x,x\rangle\geq 0, =0\leftrightarrow x=0$,
$\langle\alpha x+\beta y,z\rangle=\alpha\langle x,z\rangle+\beta\langle
y,z\rangle$, $\langle x,y\rangle=\bar{\langle y,x\rangle}$

Ekvivalentne: pozitivne definitni hermitovska bilinearni forma.
Napriklad: "defaultni": $\sum x_i\bar{y_i}$, urcity integral na spojitych
funkcich.

Norma: $\|v\|\geq 0,=0\leftrightarrow v=0$, $\|\alpha x\|=|\alpha|\cdot\|x\|
(\alpha\in\C)$, $\|x+y\|\leq\|x\|+\|y\|$
VP s normou je "normovany".
Norma dana sk. soucinem: $\|x\|=\sqrt{\langle x,x\rangle}$.

Cauchy-Schwartz pro normu ze skalarniho soucinu:
$|\langle x,y\rangle|\leq\|x\|\cdot\|y\|$.
Pres $p(t):=\|u+tv\|^2$, kde vime ze je to vsude kladne a jedno reseni atd.

Normy: $L_1: \sum|x_i|$, $L_z=(\sum_{|x_i|^p})^{1/z}$.
$L_\infty$: $\max(|x_i|)$. $L_1$ a $L_\infty$ neodpovida
zadnemu skalarnimu soucinu. V integr. funkcich: $\int_a^b f^2(x)dx$.

Vezmi $x=(x_i), y=(1)^n$, dosad do Cauchy-Schwarze: $\sum_{x_i}/n\leq\sqrt{1/n
\sum x_i^2}$

Uhel: $cos\varphi=\langle u,v\rangle/(\|u\|\|v\|)$
Z toho kosinova veta: $\|u-v\|^2=\|u\|^2+\|v\|^2-2\|u\|\|v\|\cos\varphi$

Kolmost, ortonormalni system.
Gram-Schmidtova ortogonalizace: koeficienty $a_{i,j}=\langle
v_i,w_j\rangle/\|w_j\|^2$.
Fourierovy koeficienty pro ortonormalni bazi: $x=\sum\langle x,b_i\rangle b_i$

$V$ podprostor $W$, pak:
$V^\perp$ podprostor $W$.
$\dim(V^\perp)=\dim(W)-\dim(V)$. ${V^\perp}^\perp=V$.
$\cap=\{0\}$. $\oplus=W$. $\subseteq\rightarrow\supseteq$,
$(U\cap V)^\perp=U^\perp \oplus V^\perp$.
$(U\oplus V)^\perp=U^\perp \cap V^\perp$.
Ortogonalni projekce.
