\begin{itemize}
\item Základná pojmy teorie grafů, reprezentace grafu.
\item Stromy a základní vlastnosti, kostra grafů.
\item Eulerovské a hamiltonovské grafy.
\item Rovinné grafy, barvení grafů.
\end{itemize}

\begin{understood}
Kružníce $C_n$, cesta $P_n$.
\end{understood}
Sled = vrchol -- hrany, tah = sled bez opakovani hran.
Slabá, silná souvislost pro orientované grafy:
cesta jedním směrem, oběma.

Vrcholova $n$-souvislost je silnejsi nez hranova.

TODO: jake jsou souvislosti v klikach?

Most, artikulace (kriticky vrchol).
Blokovy graf: artikulace a tak. Blokovy graf je strom.

Hranove pokryti: kazdy vrchol v aspon jedne.
Disjunktni hrany: parovani.
$k$-faktor: podgraf s konstantnim stupnem $k$.
Princip sudosti: $\sum\deg(v)=2|E|$
Skóre: posloupnost stupňů.

Věta o skóre: $D$ je SETRIDENE skore grafu prave kdyz $D'$
je skode grafu, kde $d_i'=d_i$ pro $i<n-d_n$,
$d_i'=d_i-1$ pro $i\geq n-d_n$.

Metrika: nejkratsi cesty.
+- hrana, +- vrchol, deleni hrany.

Nakresleni.
\begin{understood}
Matice sousednosti: $V\times V$.
Z ni: matice vzdalenosti.
\end{understood}

Laplaceova matice: $\deg u$ pro $u=v$, $-1$ pro $\{u,v\}\in E$,
jinak 0. Hodi se k pocitani poctu koster.

Matice incidence: radky jsou vrcholy, sloupce hrany.
-1 kdyz zacina, +1 kdyz konci hrana, 0 jinak.

\begin{understood}
Seznam sousedu. Seznam hran.

Strom: souvisly graf bez kruznice.
\end{understood}

Veta: pocet stromu je $n^{n-2}$.

\begin{understood}
To same: souvisly/acyklicky a $e=v-1$.
Maximalni bez kruznic. Minimalni souvisly.
Jednoznacne cesty.
\end{understood}

TODO: Pocet koster a Laplaceova matice

Eulerovske grafy.

Vrcholove X-souvisly $\leftrightarrow$ X vrcholove disjunktnich cest.
2-souvisly $\leftrightarrow$ vytvoritelny z $K_3$ delenim a pridavanim hran.

Oblouk: proste zpojite zobrazeni z $(0,1)\imp\R^2$.
Stena grafu.

Jordanova veta o kruznici: topologicka kruznice deli rovinu na vnitrek a vnejsek
se spolecnou hranici (uzavr. krivka neprotinajici sebe sama = oblouk, co mu
splyvaji konce).

Kuratowskeho veta: rovinnost == nema podgraf isomorfni deleni $K_{3,3}$, $K_5$.

Euleruv vzorec: $|V|-|E|+s=2$.

2-souvisly RG ma hranice libovolne stejny kruznice.

Maximalni pocet hran RG: $|E|\leq 3|V|-6$, rovnost pro maximalni RG.
Kdyz nema trojuhelnik, musi by $|E|\leq |V|-4$.

Multigraf (nasobne hrany).
Geometricky dual grafu.
$\chi$ = vrcholova barevnost.
Dual RG je RG.
Barveni mapy na barveni grafu.
RG maji chiralitu $\leq 5$.

Dijkstra: $O(n\log n+m)$ pres Fib haldy.
BFS, DFS, Kruskaluv algoritmus, Jarnikuv algoritmus, toposort.

Mengerova veta, dualita toku a rezu
