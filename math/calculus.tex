\begin{itemize}
\item Reálné funkce jedné proměnné
\item Spojitost, limita funkce v bodě (vlastní i nevlastní).
\item Některé konkrétní funkce (polynomy, racionální lomené funkce, goniometrické a cyklometrické funkce, logaritmy a exponenciální funkce).
\item Derivace: definice a základní pravidla, věty o střední hodnotě, derivace vyšších řádů.
\item Některé aplikace (průběhy funkcí, Taylorův polynom se zbytkem).
\end{itemize}

Limita funkce: $\a\eps\e\delta:x\in P(a,\delta)\imp f(x)\in B(A,\eps)$

Heineho věta: $f$ def. na nějakém $P(a,)$. $\lim_{x\imp a}f(x)=A\in\R^*
\leftrightarrow$ pro každou $\{x_n\in D(f)\}$ kde $\lim x_n=a,x_n\neq a$
je $\lim_{\infty}f(x_n)=A$

Idea důkazu: $1\imp 2$: pro $\eps$ okolí najdu $\delta$ okolí $a$, od $n_0$ dál
OK. $2\imp 1$: sporem $\e\eps\a\delta\e x\in P(a,\delta): f(x)\not\in
B(A,\eps)$. Zmenšuj $\delta$, dostávej $x$ v $P(a,\delta)$ nekonvergující
$f(x)$ k $A$ (vzdálené $\eps$), spor.

Má-li fce vlastní limitu, je na $P$ omezená.
Aritmetika limit pro $+,\cdot,/$.

$\lim f>\lim g\imp\e P$ kde $f>g$.
$f\leq g$ a $\e\lim f,\lim g \imp \lim f\leq\lim g$.
$f\leq h\leq g, \e\lim f,\lim g\imp\e\lim h, \lim h=\lim\ldots$.

Limita složené funkce: $\lim_{x\imp a}g(x)=A, \lim_{y\imp A}f(y)=B$
plus bud $f$ spojita v $A$, nebo $\e\delta>0: g(x)\neq A$ na $P(a,\delta)$
(vnitrni funkce "lokalne prosta")

Dukaz: vezmi $\eps$, pro nej najdi $\mu$ ze $\mu$-okoli $A$ je mapovane
$f(x)$ na $\eps$-okoli $B$.

$0<|x-A|<\mu \imp |f(x)-B|<\eps$. Pro $g(x)$ najdu $\delta$ pro $\mu$, ze
$0<|g-a|<\delta\imp |g(x)-A|<\mu$. Musím se vyhnout $g(x)=A$ (druha
varianta), nebo kdyz je $f$ spojita, tak to je OK.

Darbouxova veta o nabyvani mezihodnot: $f$ spojita na $<a,b>$, $f(a)<f(b)
\imp$ nabyva mezihodnot.

Lemma: pokud je spojita $f(x_0)>y$, tak je vetsi i na nejakem $P$
(vezmi chytre $\eps$).
Hledejme $x:f(x)=y$. Zkoumej $\{x:f(x)\leq y\}\imp$ vem supremum.
Kdyz $f(s)>y$, tak by byla vetsi i pro nejaka $x<s$, tedy neni nejmensi horni
mez. Kdyz $f(s)<y$, tak nekde $x>s$ je $f(s)$ mensi a neni to horni mez.
TODO: pochopit

$f($interval$)=$interval kdyz je tam spojita.

$f$ spojita na $<a,b>\imp$ nabyva maxima i minima (vezmi $\sup,\inf$)
Spojita funkce je na uzavrenem intervalu omezena.

$f$ spojita a rostouci/klesajici na $I\imp$ ma inverzi.

\subsection{Konkretni funkce}
$\existsone\exp$: $\exp(x+z)=\exp(x)\exp(z)$, $\exp(x)\geq 1+x$.

$\lim_{x\imp 0}\frac{\exp x-1}{x}=1$

$\log:=\exp^{-1}$. $\lim_{x\imp 1}\frac{\log x}{x-1}=1$
Obecna mocnina definovana pres exponencialu.

$\existsone s$ licha a $c$ suda,
$s>0$ na $(0,1)$, $s(1)=0$, $s(x+y)=s(x)c(y)+c(x)s(y)$,
$c(x+y)=c(x)c(y)-s(x)s(y)$

$\sin,\cos,\tan,\cotg$ odtamtud, $\arcsin,\arccos,\arctg,\arccotg$.
Plati: $\lim_{x\imp 0} \frac{\arcsin x}{x}=1,
	\lim_{x\imp 0} \frac{\arccos x}{\sqrt{1-x}}=\lim_{x\imp 0}\frac{\arctg
	x}{x}=1$

Derivace $\lim\frac{f(a+\Delta)-f(a)}{\Delta}$, zleva a zprava.
Ma-li $f$ derivaci, je tam spojita ($\lim_{x\imp a}[f(x)-f(a)]=f'(x)\cdot 0$).

$(f+g)'=f'+g'$ je-li definovano. $f/g$ spojita v $\imp (fg)'=f'g+fg'$.

Derivace slozene funkce: $f$ derivace v $y_0$, $g$ derivace v $x_0$,
$g$ spojita v $x_0$, $y_0=g(x_0) \imp (f\cdot g)'(x_0)=f'(y_0)\cdot g'(x_0)=
f'(g(x_0))\cdot g'(x_0)$.
Dukaz je pracny: upravime $[f(g(x))-f(g(a))]/(x-a)$, rozlozime na $A:
g(x)=g(a)$, $B: g(x)\neq g(a)$. TODO

Derivace inverzni funkce. TODO dukaz.

Derivace a rust/klesani. Nutna podminka lok. extr: $f'=0$ nebo $\not\exists$.

Rolleova věta: $f$ spojita na $<a,b>$, $\e f'$ na $(a,b)$, $f(a)=f(b)\imp
f'(c)=0$ (ma-li lokalni extrem, hotovo, jinak je to konstanta).
Lagrangeova o střední hodnotě: zobecni.

Cauchyova věta o střední hodnotě: $f,g$ spoj. na $<a,b>$ a na $(a,b) \e$
derivace a $g'$ vlastni $\neq 0\imp$
$\e c\in(a,b): \frac{f(b)-f(a)}{g(b)-g(a)}=\frac{f'(c)}{g'}$
Intuice: tecna v parametrickem plotu $[f,g]$ je spojnice.
Dosad do Rolleovy vety $f(x)+\alpha g(x)\imp\e c: \frac{f'(c)}{g'(c)}=\alpha$

L'Hospitalovo pravidlo: $a\in\R^*$, $f,g$ na $P(a,\delta)$ s vlastni derivaci,
$g'\neq 0$, $\e\lim_{x\imp a}\frac{f'(x)}{g'(x)}$ a
bud $(\imp a): \lim f(x)=\lim g(x)=0$, nebo $\lim g(x)=\pm\infty$.
Pak $\e\lim f/g=\lim f'/g'$.

Limita derivaci: $f$ spojita zprava, pro $a\in\R \e\lim_{x\imp
a^+}f'(x)=A\in\R^*$. Pak $f'_+(a)=A$.
TODO: to asi neni uzitecny

Derivace a monotonie spojite $f$.

Inflexe: prstencove okoli vlevo pod a vpravo nad tecnou.
$f''(x)\neq 0\imp$ neni inflexe.
Spojita $f'$ a $f''(<x)>0$ a $f''(>x)<0\imp$ inflexe.
(ryzi) konvexita/konkavnost.

$f$ konvexni na $I$, $a\in Int(I)\imp\e f'_\pm(a)\in\R$ (konvexnost implikuje
jednostranne derivace)
Konvexnost $\imp$ spojitost.

Asymptota $ax+b$: $f$ definovana na okoli $\pm\infty$,
$\lim_{x\imp\pm\infty}(f(x)-ax-b)=0$.
$\leftrightarrow\lim f(x)/x=a \wedge \lim f(x)-ax=b$

\subsection{Aplikace}

Vysetreni prubehu: $D$, obor spojitosti, symetrie/periodicita,
limity v krajich $D$, derivace a intervaly monotonie, lok. a glob. extremy,
obor hodnot, druha derivace a kon/kon, inflexni body, asymptoty, graf.

Taylorak: $T_n^{f,a}(x)=\sum f^{(i)}(a)\frac{(x-a)^n}{i!}$.
Jednoznacny $\lim_{x\imp a}\frac{f(x)-T_n^{f,a}(x)}{(x-a)^n}=0$ stupne $\leq n$.

TODO: obecny tvar zbytku a dukaz, Lagrangeuv a Cauchyho tvar zbytku,
Newtonova metoda.
