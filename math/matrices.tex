\begin{itemize}
\item Matice a jejich hodnost.
\item Operace s maticemi a jejich vlastnosti.
\item Inverzni matice.
\item Regularni matice, ruzne charakteristiky.
\item Matice a linearni zobrazeni, resp. zmeny souradnych soustav.
\end{itemize}

Radkovy prostor, sloupcovy prostor, kernel. Rank = \# nezavislych sloupcu.
$rank(A)=dim R(A)=dim S(A)=rank(A^T)$ z RREF tvaru po Gaussove eliminaci.
$dim(Ker A)+rank(A)=n$. Matice jsou vektorovy prostor.

$rank(AB)\leq\min\{rank(A),rank(B)\}$. Inverzni matice je oboustranne inverzni.

$rank(RA)=rank(A)$ pro regularni $A$ (aplikuj nerovnost na $R^{-1}\cdot(RA)=A$).

Souradnicovy vektor v bazi. Matice linearniho zobrazeni jde napsat jako
$f(x_j)=\sum_{i=1}^m \alpha_{ij}y_i$ pro $x_i,y_i$ baze. Tahle matice
se zapise jako $\ _{B}[f]_{B'}$. ($f$, kdyz jde z $B$ do $B'$).

$\ _{B}[f]_{B'}$ je sestavena se sloupcu $([f(x_1)]_{B'}\ldots [f(x_n)]_{B'})$,
coz jsou sour. vektory v bazi $B'$.
Mezi kazdymi dvema bazemi jde prejit matici prechodu.
