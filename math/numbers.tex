\begin{itemize}
\item Vlastnosti přirozených, celých, racionálních, reálných a komplexních
	čísel
\item Posloupnosti a řady čísel
\item Cauchyovské posloupnosti
\end{itemize}

\paragraph{$\R$, $\N$}
Axiomaticky: $(0,1,+,\cdot)$ je komutativní těleso, $\leq$ je uspořádání co se
chová rozumně s $+$ a $\cdot$ (tj. posun o číslo, násobení kladným číslem).
Netrivialita ($0\neq 1$).

TODO: co znamena "rozumne"? co je zac ta vazba?

Úplnost: každá neprázdná shora omezená podmn. má supremum, tj. nejnižší horní
mez: $\forall x\in X: x\leq s$ a $\forall s'<s\in\R \exists x\in M: x>s'$.
(Axiom suprema.)

$\N$ z $\R$: induktivní := $1\in S, \forall x\in S: (x+1)\in S \longrightarrow$
$\N$ průnik induktivních podmn. $\R$.

$\N$ je induktivní. $\forall n\in\N:n\geq 1$,
$n\neq 0\rightarrow\exists m\in N: n=m+1$, $m<n\rightarrow m+1\leq n$,
každá neprázdná množina má nejmenší prvek.

Archimedova vlastnost: $\forall x\in\R\exists n\in\N:x<n$.

Peanovy axiomy: $\exists 0, S(x)$. $\notexists x: S(x)=0$. $S(x)\neq S(y)$.
$\varphi(0)\wedge \varphi(x)\rightarrow\varphi(S(x))\longrightarrow\forall\varphi$.
Konstrukce na množinách: $S(a)=a\cup\{a\},0=\emptyset$ (axiom nekonečna).

\subsection{$\Z$, $\Q$}
$\Z$: $\pm\N + 0$, $\Q=\{p/q:p\in\Z,q\in\N\}/\sim$.

$\forall x\in\R\exists [x]\in\Z: x-1<[x]\leq x$

Hustota: $a,b\in\R\rightarrow\exists x\in\Q: a<x<b$; husté je i
$\R\smallsetminus\Q$.

Existence odmocniny: $\existsone y\geq 0\in\R: y^n=x$

$\C$: $i\gets\sqrt{-1}:i^2+1=0$. Algebraický tvar.
$|z|=\sqrt{a^2+b^2}$. $\bar{zw}=\bar{z}\bar{w}$, $|z|^2=z\bar{z}$,
$z^{-1}=\bar{z}/|z|^2$

Goniometrický tvar: $|z|(\cos\varphi+i\sin\varphi)$. Dělící a násobící vzorce,
"Moiverova věta" na mocnění.

\subsection{Posloupnosti a limity}
Omezenosti, (ryzí) monotónnosti.
Vlastní limita. Když ji nemá, diverguje. Existuje nejvýš 1.
Konvergentní posloupnosti jsou omezené.
Vybrání podposloupnosti zanechá limitu (existuje-li).
Nevlastni limita $\pm\infty$. $\R^*$.

Aritmetika limit: $+,\cdot$, $b_i\neq 0,B\neq 0\rightarrow/$

TODO: prove?
TODO: dokazat pro deleni

Pro $\R^*$ funguje je-li výsledek definován.

$\sup\emptyset:=-\infty,\inf\emptyset:=\infty$

Policajti. Soucin mizejici a omezene mizi.
Podil kladne a mizejici nezaporne jde k $\infty$.

Kazda monotonni posloupnost ma limitu: omezena ma vlastni, jinak nevlastni.
$\lim\sup$ je limita horních mezí prefixů, obdobně $\lim\inf$ ($\pm\infty$
jsou-li posloupnost prefixů neomezené). Vždy existují.
$\lim=a\in\R^*\leftrightarrow\lim\sup=\lim\inf=a$

Hromadný bod: limita vybr. posloupnosti. $\lim\sup=\max H,\lim\inf=\min H$

\subsection{Cauchyovské posloupnosti}
Bolzano-Weierstrass: omezená má konvergentní podposloupnost.
Jsou-li $a_n$ v komp. intervalu $[a,b]$, je tam i limita vybr. podposl.
Můžu třeba vzít $\lim\sup$ a zmenšovat interval.

Cauchyovská (Bolzano-Cauchy podmínka):
$\forall\varepsilon\exists n_0\forall m,n\geq n_0: |a_n-a_m|<\varepsilon$

Lemma: podposloupnost C posloupnosti konverguje $\rightarrow$ C posl.
konverguje.
Bolzano-Cauchy: konvergentní $\leftrightarrow$ Cauchyovská. $\rightarrow$
hned, pro $\leftarrow$ staci omezenost Cauchyovske posl., Bolzano-Weierstrass
a lemma. Omezenost: vem treba $\epsilon=1$.

TODO: nejake ty rady, jejich konvergence, srovnavaci kriteria, atd.?
