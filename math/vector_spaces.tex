\begin{understood}
	\begin{itemize}
	\item Grupa, teleso.
	\item Zakladni vlastnosti vektorovych prostoru, podprostory,
		generovani, linearni zavislost a nezavislost.
	\item Veta o vymene.
	\item Konecne generovane vektorove prostory, baze.
	\item Linearni zobrazeni.
	\end{itemize}
\end{understood}

Definice vektoroveho prostoru: $(T,+,\cdot)$ teleso, $+$, $\cdot$
Axiomy: $+$ komutativni grupa, $\cdot$ asociativni, $1u=u$, $(a+b)u=au+bu$,
$(ab)u=au+bu$, $a(u+v)=au+av$.

Podprostor: uzavreny na $\cdot,+$. Prunik podpr. je podpr.
Lin. obal: prunik $\supseteq$ podpr.; system generatoru.
$L=\{linearni\ kombinace\}$.
Spojeni.
Linearni nezavislost.

Baze: linearne nezavisla co to spanuje. Vyjadreni k bazi: $[u]_X$.
Lemma o vymene: $v=\sum a_i v_i:: a_i\neq 0\imp span(v_1,\ldots
v_{i-1},v,v_{i+1},\ldots)=V$.
Steinitzova veta o vymene (pro kon. generovane): nezavisly system
jde rozsirit pridanim par vektoru z baze na system generatoru.
Kvuli tomu muzu definovat dimenzi. $\dim A+\dim B=\dim(A\cap B)+\dim(A\cup B)$.

Linearni zobrazeni: $f(ax+by)=af(x)+bf(y)$.
$ker=f^{-1}[\{0\}]$. $\dim(\Ker f)$ = hodnost zobrazeni.
$f$ je proste $\leftrightarrow \Ker f=\{0\}$.
$dim X=dim Y$ a $f$ proste $\leftrightarrow f$ bijekce, $f^{-1}$ linearni.

$\dim(\Ker f)+\dim(\Im f)=\dim(A)$.

Obraz baze jednoznacne urcuje zobrazeni.
Isomorfismus $\leftrightarrow$ bijekce.
Proste LZ zachovava nezavislost.
Surjektivni LZ zachovava vlastnost "byt system generatoru".
Mnozina LZ je vektorovy prostor.
$\dim L(V,W)=mn$
