\begin{itemize}
\item Primitivní funkce, metody výpočtu
\item Určitý (Riemannův) integrál, užití určitého integrálu
\item Vícerozměrný integrál a Fubiniho věta
\end{itemize}

$F$ je primitivní, když $F'=f$. $G(x)=F(x)+C$. Linearita.
Spojité funkce mají primitivní funkce.

Substituce: 1. $\int f(\varphi(t))\varphi'(t)dt=F(\varphi(t))+C$.

TODO: TOHLE MUSIM UMET
2. $G(t)=\int f(\phi(t))\phi'(t)dt$ na $(\alpha,\beta)$, $\phi'\neq 0$ a ma
	$\phi'$, $\phi((\alpha,\beta))=(a,b) \imp \int f(x) dx=G(\phi^{-1}(x))$

Per partes: $\int u'v=uv-\int uv'$

Polynomy jdou dělit. Mají $k$ komplexních kořenů.
Polynomy s reálnými koeficienty mají komplexně združené kořeny.
Polynomy s reálnými koeficienty jdou rozdělit na kořenové činitele
a polynomy stupně 2. Přes tohle si převedu zlomek polynomů na parciální zlomky.

Riemannův integrál: dělení intervalu konečně mnoha dělícími body s nějakou
jemností. $S(f,D)=\sum \sup \cdot (x_j-x_{j-1}$ a podobne $s(f,d)$.
Horni Riemannuv integral: $\inf\{S\}$, dolni: $\sup\{s\}$.
Ma Riemannuv integral: rovna se horni a dolni.

Je-li funkce omezena, pak ma integral prave kdyz pro kazde $\eps>0$
existuje deleni, ze $S(f,D)-s(f,D)<\eps$.
Monotonie a linearita Riemannova integralu.
Spojite funkce maji Riemannuv integral. (Dokonce: omezena a spojita s vyjimkou
konecne mnoha $\imp$ existuje integral)
Aditivita.
Omezena a monotonni funkce $\imp$ Riemannovsky integrovatelna.
Riemannuv integral funguje jako primitivni funkce.

Zakladni veta analyzy: $f$ spojita na $<a,b>$, $F$ primitivni k $f$ na
$(a,b)\imp\e$ vlastni limity $\lim_{a\imp a^+}F(x)$ a $\lim_{x\imp b^-}F(x)$
a plati: $\int_a^b f=\lim_{x\imp b-}F(x)-\lim_{x\imp a+}F(x)$.
Newtonuv integral: kdyz mame primitivni funkci, tak je to presne tenhle rozdil
limit. Riemannovsky a Newtonovsky integral se rovnaji. Riemannovska a
Newtonovska integrovatelnost spolu nesouvisi.
Nekonecne integraly jsou vzdycky Newtonovske.

Délka křivky: $\int_a^b \sqrt{1+(f'(x))^2}dx$.
Objem rotačního tělesa: $V=\pi\int_a^b f(x)^2 dx$.

$f$ spojita, nezaporna a nerostouci na $<n_0-1,\infty> \imp$
$\sum_n=1^\infty f(n)$ konverguje $\leftrightarrow (N)\int_{n_0}^\infty
f(t)dt<\infty$.

Kompaktni interval ve vice rozmerech, rozdeleni po slozkach.
Vicerozmerny Riemannuv integral $\leftrightarrow\a\eps>0\e D:
S(f,D)-s(f,D)<\eps$.

$|\int_J f|\leq \int_J |f|$.
Fubiniho veta: spojitou funkci muzu vicerozmerne integrovat v libovolnem
poradi: $\int_J f(\vec{x},\vec{y})
d\vec{x}\vec{y}=\int_{J'}(\int_{J''}f(\vec{x},\vec{y})d\vec{y})d\vec{x}=\int_{J''}(\int_{J'}
f(\vec{x},\vec{y})d\vec{x}) d\vec{y}$

Parcialni derivace: limita o epsilon-krok.
Jacobiho matice $\vec{f}:D\imp\R^n$: $\left(\frac{\partial f_i}{\partial
x_j}(a)\right)^{n,m}_{i,j=1}$. Determinantu se rika Jakobian.
Regularni zobrazeni: nezaporny Jakobian ve vsech bodech.

% Vynechavam veta o substituci pro vicerozmerne funkce.
