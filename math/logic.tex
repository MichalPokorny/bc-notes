\begin{itemize}
\item Jazyk, formule, sémantika, tautologie.
\item Rozhodnutelnost, splnitelnost, pravdivost, dokazatelnost.
\item Věty o kompaktnosti a úplnosti výrokové a predikátové logiky.
\item Normální tvary výrokových formulí, prenexní tvary formulí predikátové logiky.
\end{itemize}

Jazyk: $x_i$, $\vee\wedge$, $()$, $\a\e$, $R$, $f$, $(=)$ optionally

Logický systém: jazyk + axiomy + odvozovací pravidla.

Termy: individua, formule: tvrzení

VL: prvovýroky, spojky, závorky

Stačí $\neg,\imp,\a$.
Definice formule, termu.

Volný, vázaný výskyt; volná proměnná
Otevřená: nemá vázanou proměnnou, uzavřená/sentence: nemá volnou proměnnou.

Pravdivostní ohodnocení: ohodnotí prvovýroky, z něho rekurzí pravdivostní
hodnota formulí.
Realizace jazyka v PL: individua v $M$, relace, funkce

Definice tautologie (PL): pravdivá při liboovolném ohodnocení proměnných
($\vDash A$)
Pravdivá formule při ohodnocení: rekurzivně definované.
Model, tautologický důsledek $T\vDash$ (každý model $T$ to modeluje taky).

Formální systém (Hilbertův kalkul): modus ponens, plus axiomy:
1) $A\imp(B\imp A)$, 2) $(A\imp(B\imp C))\imp((A\imp B)\imp(A\imp C))$,
3) $(A\imp B)\imp(\neg B\imp\neg A)$

Substituce (termu za proměnné). Instance formule.

Formální systém VL: přidej axiomové schema specifikace $(\a x)A\imp A_x[t]$,
přeskoku $(\a x)(A\imp B)\imp(A\imp (\a x)B)$ když $x$ nemá volný výskyt v $A$,
pravidlo generalizace: $A \longrightarrow (\a x) A$.
S rovností: přidej symbol a další 3 axiomy. TODO: které?

Pravidlo zavedení $\a$: $\vdash A\imp B$ a $x$ nemá VV v $A$, tak $\vdash A\imp
(\a x)B$.
Zavedení $\e$: $\vdash A\imp B$ a $x$ nemá VV v $B$, tak $\vdash (\e x)A\imp B$.
Distribuce: když $\vdash A\imp B$, tak $\vdash (Qx)A\imp (Qx)B$.

Důkaz: Hilbertův kalkul, nebo tablo metoda.

Věta o dedukci: $T\vdash A\imp B$ právě když $T,A\vdash B$.
($A$ musí být uzavřená -- indukce: pravidlo generalizace??? wtf?)

Mnozina formuli VL je splnitelna prave kdyz je bezesporna.

Rekurzivní funkce: algoritmicky vyčíslitelné. Rekurzivní množina: ta, co má
rekurzivní $\chi$ fci.

Ve spocetnem jazyku jdou ocislovat formule. Thm(T) = kody dokazatelnych formuli.
Rozhodnutelna teorie: Thm(T) je rekurzivni.

Churchove veta o nerozhodnutelnosti predikatove logiky: kdyz mam 1 konstantu,
1 fcni symbol arity >0 a "pro kazde prirozene cislo spocetne mnoho predikatovych
symbolu" (wtf?), pak je mnozina kodu uzavrenych pravdivych formuli
nerozhodnutelna.

Popisy aritmetiky:

Robinsonova aritmetika: $S(x)\neq 0$, $S(x)=S(y)\leftrightarrow x=y$,
$x\neq 0\imp (\e y)(x=S(y))$, $x+0=x$, $x+S(y)=S(x+y)$, $x\cdot 0=0$,
$x\cdot S(y)=(x\cdot y)+x$, $x\leq y\leftrightarrow(\e z)(z+x=y)$

Peanova aritmetika: nema axiom 3 ($x\neq 0\imp(\e y)(x=S(y))$), plus schema
axiomu indukce: $(\varphi(0)\wedge (\a x)(\varphi(x)\imp\varphi(S(x))))\imp(\a
x)(\varphi(x))$

Úplná aritmetika: axiomy jsou všechny uzavřené formule platné ve standardním
modelu $\N$.

$Q$ (Robinsonova) $\subseteq P$ (axiomy jsou rekurzivni mnozina) $\subseteq
Th(\N)$ (neni rekurzivne axiomatizovatelna)

Churchova veta o nerozhodnutelnosti aritmetiky: každé bezesporné rozšíření
Robinsonovy aritmetiky $Q$ je nerozhodnutelna teorie. (TODO: WTF?!)

Godel-Rosserova věta o neúplnosti aritmetiky: žádné bezesporné a rekurzivně
axiomatizovatelné rozšíření $Q$ není úplná teorie.

TODO: tyhle rozhodnutelnosti/axiomatizovatelnosti radsi jeste jednou projit.

Věta o kompaktnosti a úplnosti výrokové a predikátové logiky.

O korektnosti výrokové logiky: $T\vdash A \Rightarrow T\vDash A$ (co je
dokazatelné, je pravdivé).
Má-li teorie model, je bezesporná.
Věta o spoustě věcí: sporná právě je-li dokazatelný spor. $T,\neg A$ je sporná
právě když $T\vdash A$. Kdyz je $T$ maximalni bezesporna teorie, tak
$T\vdash A \Leftrightarrow A\in T \Leftrightarrow T,A$ je bezesporná.
Ohodnoceni $e$ takové $e(p)=1 \Leftrightarrow p\in T$ pro každou výrokovou
proměnnou $p$, je jediný model $T$.

Bezesporná teorie má maximální bezesporné rozšíření.

Kompaktnost: teorie je pravdivá právě když je pravdivá každá konečná podmnožina.

Úplnost: $T\vdash A\Leftrightarrow T\vDash A$ platí pro každou teorii $T$ a její
formuli $A$. Důsledkem je bezespornost výrokové logiky a dokazatelné v ní jsou
právě tautologie. WTF.

V PL má každá bezesporná teorie model kardinality nejvýš $\|L\|$.
Formule PL je dokazatelná právě když je pravdivá.
Teorie má model (je splnitelná) právě když každá konečná část má model.

Normální tvary:
DeMorganovy zákony: $\neg(A\wedge B)\leftrightarrow(\neg A\vee\neg B)$;
	$\neg(A\vee B)\leftrightarrow(\neg A\wedge\neg B)$

Hornovská klauzule: nejvýš jeden pozitivní literál.

DeMorganovými zákony a distributivitou jde sestrojit CNF, DNF formule.

Prenexní tvary: kvantifikátory, vevnitř otevřené jádro.
Skolemizace: náhrada existenčních kvantifikátorů funkcí, otevření axiomů.
