\begin{itemize}
\item Casova slozitost algoritmu, slozitost v nejhorsim a prumernem pripade.
\item Tridy slozitosti P a NP, prevoditelnost, NP-uplnost.
\item "Rozdel a panuj" - aplikace a analyza slozitosti, dynamicke programovani.
\item Binarni vyhledavaci stromy, vyvazovani, haldy.
\item Hashovani.
\item Sekvencni trideni, porovnavaci algoritmy, prihradkove trideni, tridici site.
\item Grafove algoritmy - DFS, BFS, souvislost, topologicke trideni, nejkratsi cesta, kostra grafu, toky v sitich.
\item Tranzitivni uzaver.
\item Algoritmy vyhledavani v textu.
\item Algebraicke algoritmy - DFT, Eukliduv algoritmus, RSA.
\item Aproximacni algoritmy.
\end{itemize}

TODO: vsechno ostatni

$$f(n)=O(g(n)): \e c>0 \e n_0 \a n>n_0: 0\leq f(n)\leq c\cdot g(n)$$
$$f(n)=\Omega(g(n)): \e c>0 \e n_0 \a n>n_0: 0\leq c\cdot g(n)\leq f(n)$$
$$f(n)=\Theta(g(n)): \e c_1,c_2>0 \e n_0 \a n>n_0: 0\leq c_1\cdot g(n)\leq
f(n)\leq c_2\cdot g(n)$$

Amortizovana slozitost: prumerny cas na operaci. Agregace: spocitam $T(n)$ na
celou posloupnost, cas bude $T(n)/n$.
Ucetni: od operace neco vyberu, z toho zaplatim jeji provedeni, zbytek dam na
ucet. Je-li operace drazsi nez obnos, vybereme z uctu. Zustatek nezaporny, obnos
= amortizovana cena.

Problemy: rozhodovaci, optimalizacni.
Polynomialni preveditelnost rozhodovacich problemu: kladne prave kdyz je kladne,
deterministicky preveditelne v P case.

3-SAT: CNF klauzule delky 3.
3-COL: 3-colorable graph?
Nezavisla mnozina: existuje nezavisla mnozina velikosti K?

3-SAT -> 3-COL.
3-COL -> nezavisla mnozina (trivialni -- 3 kopie).
nezavisla mnozina -> 3-SAT.
Hamiltonovska kruznice, batoh, obchodni cestujici (rozhodovaci).

Rozdel a panuj: $T(n)$ na zpracovani, $D(n)$ na rozdeleni na velikosti $n/c$,
$S(n)$ na sjednoceni uloh velikosti $n/c$ na ulohu velkou $n$.
Rovnice: $T(n)=D(n)+aT(n/c)+S(n)$, $\O(1)$ pro triviality.

Substitucni metoda: uhodni a dokaz.

Master theorem: $a\geq 1, c>1, d\in\R\geq 0, T$ neklesajici, pro $n=c^k$:
$T(n)=aT(n/c)+\Theta(n^d)$. Pak kdyz $a\neq c^d$, tak
$T(n)=\Theta(n^{\max\{\log_c a, d\}})$, jinak $T(n)=\Theta(n^d \log_c n)$.

Karatsuba algoritmus: $\Theta(n^{\log_2 3})$.
Strassenuv (na matice): $\Theta(n^{\log_2 7})$

$k$-ty nejmensi v linearnim case: mediany lzimedianu (oddelim aspon 3/10,
hezky graficky dukaz).

Red-black: (externi uzel/koren jsou cerne), cerveny vrchol ma cerne
syny a otce, cesta od vrcholu k listum ma stejne cernych uzlu.

TODO: insertion, deletion

AVL stromy: rozdil vysky leveho a praveho je $\leq 1$.
Velikost roste pres Fibonacciho posloupnost.
Vyvazovani pres rotace.

TODO: insertion, deletionstromy

Binarni halda. Fibonacciho halda: amort. O(1) na Insert, DecreaseKey,
odebrani minima O(log n).

TODO: na wiki je algoritmus na optimalni binarni vyhledavaci stromy

TODO: detaily hashovani (ocekavane pocty testu atd.)

Univerzalni hashovani: treba $h_{a,b}(x)=((ax+b)\bmod |U|) \bmod m$
Perfektni hashovaci funkce

\subsection{Sekvencni trideni, porovnavaci algoritmy, prihradkove trideni,
tridici site}
Selection sort, insertion sort, bubble sort, heap sort, merge sort, quicksort
($\Theta(n^2)$ worstcase, idealne i prumerne $\Theta(n\log n)$).

Radix sort, counting sort.

Tridici site: bitonicka posloupnost: po spojeni do cyklu ma 2 monotonni useky.

Komparator.

Tridici sit. Bitonicke trideni: ma hloubku $\O(\log^2 n)$.

TODO: tridici sit -- bitonicke trideni

\subsection{Grafove algoritmy - DFS, BFS, souvislost, topologicke trideni,
nejkratsi cesta, kostra grafu, toky v sitich}

Toposort: najdi vrchol kam nevedou hrany, prirad nejvyssi volne cislo, odeber
ze seznamu (pry to da $\Theta(n(m+n))$).
Rychly algoritmus: DFS, vem je v poradi klesajicich casu opusteni.

TODO: hledani nejkratsi cesty

Hledani kriticke cesty: najdi topousporadani, v nem relaxuj vsechny vrcholy.

Dijkstruv algoritmus (v nezapornem ohodnoceni hran):
relaxuj nejblizsi vrchol; bude trvat $O((m+n) log n)$.

Bellman-Fordův algoritmus: funguje i s nezápornými hranami.
Pro všechny vrcholy: relaxuj přes všechny hrany.
Pak zkontroluj, že tam nejsou záporné cykly.

Nejkratsi cesty pro vsechny dvojice: Floyd-Warshall. 

\subsection{Tranzitivni uzaver}
TODO

\subsection{Algoritmy vyhledavani v textu}
TODO

\subsection{Algebraicke algoritmy - DFT, Eukleiduv algoritmus, RSA}

Vyhodnoceni polynomu v bodech $\omega^0\ldots\omega^{n-1}$, kde
$n$ je stupen polynomu BUNO $2^k$. Jsou to mocniny komplexni odmocniny.
Da se v $\Theta(N\log N)$ vyhodnotit, diky tomu ze to rozdelime na
$P(x)=S(x^2)+xL(x^2), P(-x)=S(x^2)-xL(x^2)$. Inverzni FFT se provede
diky tomu, ze FFT je linearni zobrazeni, $\Omega_{ij}=\omega^{ij}$,
$\Omega^{-1}=\frac{1}{n}\bar{\Omega}$, takze jenom podel $n$ a komplexne sdruz.

TODO: RSA, aproximacni algoritmy

\subsection{Aproximacni algoritmy}
