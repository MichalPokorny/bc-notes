\begin{itemize}
\item Casova slozitost algoritmu, slozitost v nejhorsim a prumernem pripade.
\item Tridy slozitosti P a NP, prevoditelnost, NP-uplnost.
\item "Rozdel a panuj" - aplikace a analyza slozitosti, dynamicke programovani.
\item Binarni vyhledavaci stromy, vyvazovani, haldy.
\item Hashovani.
\item Sekvencni trideni, porovnavaci algoritmy, prihradkove trideni, tridici site.
\item Grafove algoritmy - DFS, BFS, souvislost, topologicke trideni, nejkratsi cesta, kostra grafu, toky v sitich.
\item Tranzitivni uzaver.
\item Algoritmy vyhledavani v textu.
\item Algebraicke algoritmy - DFT, Eukliduv algoritmus, RSA.
\item Aproximacni algoritmy.
\end{itemize}

TODO: vsechno ostatni

\subsection{DFT}

Vyhodnoceni polynomu v bodech $\omega^0\ldots\omega^{n-1}$, kde
$n$ je stupen polynomu BUNO $2^k$. Jsou to mocniny komplexni odmocniny.
Da se v $\Theta(N\log N)$ vyhodnotit, diky tomu ze to rozdelime na
$P(x)=S(x^2)+xL(x^2), P(-x)=S(x^2)-xL(x^2)$. Inverzni FFT se provede
diky tomu, ze FFT je linearni zobrazeni, $\Omega_{ij}=\omega^{ij}$,
$\Omega^{-1}=\frac{1}{n}\bar{\Omega}$, takze jenom podel $n$ a komplexne sdruz.

TODO: RSA, aproximacni algoritmy
