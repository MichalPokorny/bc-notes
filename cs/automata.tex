\begin{itemize}
\item Chomského hierarchie, třídy automatů a gramatik, determinismus a
nedeterminismus.
\item Uzávěrové vlastnosti tříd jazyků.
\end{itemize}

Konečný automat: abeceda, přechodová funkce, stavy, koncové stavy.

Nerodova veta: regulární jazyk $\leftrightarrow \e$ prava kongruence konecneho
indexu $\sim$ t.z. $L$ je sjednoceni nejakych trid rozkladu $X/\sim$.

Prava kongruence: ekvivalence na $X^*$ t.z. $u\sim w\rightarrow uw\sim vw$.

Pumping lemma: pro regularni jazyk $\e n\in\N \a z\in L (|z|\geq n)$ lze
psat jako $l=uvw$, kde $|uv|\leq n$, $|v|\geq 1$ a $\a i: uv^i w\in L$.

Homomorfismus $\imp$ ekvivalentni automaty. Redukt: jednoznacny
ekvivalentni automat bez nedosazitelnych a ekvivalentnich stavu.

Operace: mnozinove, zretezeni, iterace, otoceni, kvocienty ($L_2\\ L_1=\{u:uv\in
L_1, u\in L_2\}$ levy, pravy $/: v\in L_2$), derivace (=kvocient podle jednosl.
jazyka). Reg. jazyky jsou na to uzavrene.

Trida reg. jazyku: obsahuje $\emptyset, \a x\in\Sigma: x$, uzavrena na
sjednoceni, iteraci a zretezeni. Kleene: regulární $\leftrightarrow$
rozpoznatelný konečným automatem.

Regulární výrazy: obsahuje $\emptyset, \lambda, x\in\Sigma$, uzavreno na
$+,\cdot,\ ^*$. Hodnota = regularni jazyk.

Dvoucestny automat: zaroven reknu jestli ted chci vlevo nebo vpravo.

Zasobnikovy automat: $(Q,X,Y,\delta,q_0,Z_0,F)$: $Y$ na zasobniku, $Z_0$ poc.
symbol, $\delta:Q\times (X\cup\{\lambda\})\times Y\rightarrow P(Q\times Y^*)$.
Nahradi vrchol zasobniku, ale necte pokazde vstupni symboly.
Prijimani: prazdnym zasobnikem ($F=\emptyset$) nebo koncovym stavem.

Prijem koncovym stavem a zasobnikem jde mezi sebou prevadet.
NEDETERMINISTICKY.

Prepisovaci system.
Kazda BKG jde rozpoznat prazdnym zasobnikem, NEDETERMINISTICKY.
Pro kazdy zás. automat existuje BKG.

Redukovana BKG: kazdy neterminal generuje nejake terminalni slovo,
kazdy neterminal $X$ se da nejak vygenerovat ze startu jako $S\imp aXy$

Greibachova normalni forma: vsechna pravidla $A\imp au$, kde $a$ je terminal
a $u$ je string neterminalu. Ke kazdemu BKJ existuje GNF, ktera prijima
$L-\lambda$.

CNF: $X\imp YZ, X\imp a$. Existuje CNF co generuje $L-\lambda$

Na GNF: 1) konverze do CNF, 2) eliminace leve rekurze
Leva rekurze $A\imp A\alpha|\beta$ generuje $\beta\alpha^2$, libovolne to
predelej.

Chomského hierarchie: typ 0 = rekurzivně spočetné, (rekurzivní nejsou v
hierarchii), typ 1 = lineárně omezené automaty (kontextové jazyky), typ 2 =
bezkontextové jazyky, (deterministické bezkontextové nejsou v hierarchii),
typ 3 = regulární.

Každá BKG jde převést na nevypouštějící (tj. nemá $X\imp\lambda$), až na lambdu.

Každá kontextová gramatika jde rozseparovat: pravidla buď $N+\imp N+$, nebo
$N+\imp T+$

KONTEXTOVA GRAMATIKA: $\alpha A\beta\imp\alpha\gamma\beta$

Determinismus a nedeterminismus.
Linearne omezeny automat: je nedeterministický.

TODO: automaty a jazyky
