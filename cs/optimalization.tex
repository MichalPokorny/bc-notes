\begin{itemize}
\item Mnohosteny, Minkowskeho-Weylova veta
\item Zaklady LP, vety o dualite, metody reseni
\item Edmondsuv algoritmus
\item Celočíselné programování
\end{itemize}

Veta o oddelovani: $C$, $D$ konvexni uzavrene, $C$ omezena (potrebuju
kompaktnost).
Minkowski-Weyl: omezeny konvexni mnohosten $\leftrightarrow$ konvexni obal
konecne mnoziny. $\rightarrow$: popadneme bodiky pro vsechny steny (nerovnosti
z minimalniho popisu). Bod vevnitr protneme primkou co popadneme za pruniky
se stenami. Indukce dle rozmeru.
$\leftarrow$: Popadnu všechny (normalizované na $\pm 1$) nerovnice platící
na $X$. Najdi body, jejichz jsou tyhle nerovnice konvexni kombinace.
Dokaz, ze body davajici mnohosten splnuji "vrcholove nerovnice".
Dokaz, ze bod mimo dokazu oddelit.

Linearni program: $\max c^T x$, $x\geq 0$, $Ax\leq b$.

Dualita: $\max c^T x, x\in\R^n, Ax\leq b \Leftrightarrow \min b^T y, y\geq 0,
A^T y=c$.

$\max c^T x, x\geq 0, Ax\leq b \Leftrightarrow \min b^T y, y\geq 0, A^T y\geq c$

$P$, $D$ ($\bar{P}$, $\bar{D}$) jsou buď obě nepřípustné, nebo jedna
neomezená druhá nepřípustná, nebo mají stejná optima.
Důkaz: TODO hnusný přes Farkasovo lemma

TODO: Edmondsuv algoritmus
TODO: celociselne programovani
