\begin{itemize}
\item Architektury databázových systémů.
\item Konceptuální, logická a fyzická úroveň pohledu na data.
\item Algoritmy návrhu schémat relací, normální formy, referenční integrita.
\item Transakční zpracování, vlastnosti transakcí, uzamykací protokoly,
zablokování.
\item ER-diagramy, metody návrhu IS.
\item Přehled SQL.
\end{itemize}

Databáze: logicky uspořádaná kolekce souvisejících dat.
Uspořádána tak, aby uměla dotazy. Obsahuje metadata (schema).

DBMS: frontend k databázi (integrita, provádění dotazů atd.)

Účel: sdílení více uživatelovi, unifikované rozhraní, znovuvyužitelnost,
bezespornost, odebrání redundance.

Schemata se dají organizovat různými modely.
"Flat model", relační model (relace v tabulkach, atributy, domeny),
hierarchicky (XML), sitovy (jako KG), objektove ("serializace objektu")

Centralizovane/distribuovane, jednouzivatelske/viceuzivatelske
Distribuovane: federativni (bez glob. rizeni). Heterogenni: unified query
interface pres nekolik DBMS.

Vice uzivatelu: uzamykani/multiversion concurrency control (commit)

Datové modelování: vyrobení schematu aplikací abstraktního datového modelu.
Konceptuální schéma: entity a vztahy. Logické schéma: implementace v databázi.
Fyzické schéma: jak je to uložené, kde a jak to poběží.

Entity-relationship diagram: kolecko = atribut, muze se rozdelit na subkolecka.
Cerne kolecko = klic. Entity jsou ctverecky, kosoctverec = relationship.
Slozeny klic: carka a klicove kolecko.

\verb#[Osoba] --N-- <MistoNarozeni> --1-- [Misto]#

Logicke schema: jak to udelame, napriklad pomoci relacnich tabulek, trid,
nebo XML. Fyzicke schema: jak se to ulozi.

Relacni algebra:
\begin{itemize}
\item Prejmenovani: $\langle R,R(A_i\rightarrow B_i)\rangle$.
\item Sjednoceni: $\langle R_1,R_1(A)\rangle\cup\langle R_2,R_2(A)\rangle$
\item Rozdil mnozin: $\langle R_1,R_1(A)\rangle - \langle R_2,R_2(A)\rangle$
\item Kartezsky soucin: $\langle R_1,R_1(A)\rangle\times\langle R_2,R_2(B)\rangle$
	(bacha -- prejmenuj stejne sloupce na $R_1 A_1, R_2 A_2,\ldots$)
\item Selekce: $\langle R(atribut=hodnota), R(A)\rangle$
\item Projekce: $\langle R[C],R(A)\rangle$: nech jenom atributy $C$ a odeber
duplikaty.
\end{itemize}
Z toho se da postavit:
\begin{itemize}
\item Prunik
\item Natural join ($\star$): spojeni pres vsechny sdilene atributy
\item Deleni: $\langle\{t:\forall s\in S:(ts)\in R\}, R_x(A-B)\}$
\end{itemize}

Relacni kalkul. Domenovy: data na urovni atributu, N-ticovy: data na urovni
N-tic.

Algoritmy navrhu relacnich schemat.
Normalni formy.

Update anomany: zmenim-li redundantni data, musim zmenit i kopie.
Insertion anomaly: pri vlozeni dat o jedne entite je potreba vlozit i jinou
entitu (napriklad \verb#(fakulta,kurz,datum zalozeni)#).
Deletion anomaly: je potreba vymazat i jinou entitu kdyz mazu jednu.

Spojeni vyslednych relaci jsou vsechny puvodni informace.

Funkcni zavislosti: $A\imp B$. Definuji se pro mnoziny atributu.
Armstrongova pravidla: $\supseteq\imp\imp$, tranzitivita, kompozice/dekompozice
($X\imp Y,Z\Rightarrow X\imp YZ$).
Superklíč: každá podmnožina, na níž schéma funkčně závisí.
(potenciální) Klíč (candidate key): minimální superklíč.
Atribut v nějakém klíči: klíčový, jiné: neklíčové.

\begin{itemize}
\item 1NF: jenom jednoduché datové typy.
\item 2NF: 1NF + existuje klic a vsechna neklicova pole jsou funkci
	\textbf{celeho} klice (ne jenom jeho casti). Priklad ne-v-2NF:
	\verb#(Name,Skill,Address)#.
\item 3NF: kazdy neklicovy atribut je zavisly \textbf{primo} na klici
	(ne jenom tranzitivne).
	Priklad ne-v-3NF: \verb#(DeptID,DeptName,ManagerID,ManagerHireDate)#
	(vadi HireDate).
\item Boyce-Codd NF: pro kazdou netrivialni $X\imp Y$ plati, ze $X$ je nadklic.
\end{itemize}

Dva pristupy: A) ER $\imp$ tabulky ktere znormalizuji, B) univerzalni schema
$\imp$ normalizace.
A hrozi rozdrobenim, B neni transparentni.

Rozdeleni na podschemata musi byt bezeztratove ($A_1\cap A_2\rightarrow A_1$
nebo $A_2$ pro rozdeleni $A$ na $A_1$, $A_2$). Plus: uzaver funkcnich
zavislosti vlevo $\cup$ vpravo musi byt stejny jako byl puvodne
(priklad spatne: \verb#Company-HQ-Altitude ==> Company-Altitude, Company-HQ#).

Dekompozice do BCNF. Zarucuje bezeztratovost, ale ne pokryti zavislosti.
(U BCNF to nemusi jit.):
vyber schema co neni v BCNF, vezmi neklicovou zavislost $X\imp Y$ (kde $X$
neni klic), odtrhni $XY$ do tabulky.

Synteza do 3NF: Zarucuje pokryti zavislosti, ale ne bezeztratovost.
1. ze zavislosti $F$ udelej minimalni pokryti $G$,
2. ze zavislosti $G$ se stejnou levou stranou udelej z kazde jedno schema,
3. zahod schemata co jsou podmnoziny jinych.
Nakonec muzu sloucit schemata s funkcne ekviv. klici ($K_1\leftrightarrow K_2$),
ale to muze porusit 3NF! Pro bezeztratovost lze pridat schema s univerzalnim
klicem puvodniho schematu.

Minimalni pokryti: 1. dekomponuj na elementarni (1 sloupec vpravo), 2. odstran
redundantni atributy (vlevo zavisejici na jinych vlevo), 3. odeber redundantni
funkcni zavislosti (tj. jsou trans. dusledkem jinych).

Referencni integrita: zabranovani vzniku chyb v relacne provazanych datech.
ON UPDATE/ON DELETE: podminka spusteni akce.
Akce: RESTRICT: nahlasi chybu. CASCADE: kaskadove smazani.
SET NULL. SET DEFAULT (pevna hodnota). NO ACTION.

TODO: kam se tohle strka?

Transakce: COMMIT/ROLLBACK prikazy.
ACID: atomicity (cela nebo nic), consistency (z konzistentniho do
konzistentniho), isolation, durability (efekty jsou nevratne ulozeny a nemuzou
se zrusit).

Rozvrh. Legalni rozvrh: objekt uzamceny kdyz k nemu pristupuju, nezamykam
jiz zamceny objekt (pockam si).
Serializovatelny/usporadatelny: jako nejaky seriovy rozvrh.
Problemy (konflikty): write-write (prepsani nepotvrzenych), read-write
(neopakovatelne cteni), write-read (cteni uncommitted dat).

Konfliktove usporadatelny: konfliktove ekvivalentni nejakemu seriovemu rozvrhu
(tj. zadne konflikty). Test: acyklicnost grafu, hrany jsou konflikty,
vrcholy jsou transakce. Slabsi nez usporadatelnost (nezohlednuje ROLLBACK,
INSERT/DELETE, atd.).
K zotavitelnosti: kazdou $T$ potvrd az kdyz potvrdis vsechny transakce co $T$
davaly data.
Deadlock. Da se detekovat.

Uzamykaci protokoly.
2PL: nejdriv zamkne vsechno, pak jenom odemyka. Pokud jsou vsechny transakce
2PL, kazdy legalni rozvrh je usporadatelny. (TODO: WTF?)
Nezarucuje zotavitelnost, bezpecnost proti kask. ruseni, uvaznuti.

Strict 2PL: uvolni zamky az po commitu. Mensi paralelismus.

Conservative 2PL: popros o zamky nez se zacnes vykonavat (prevence deadlocku).

RW-LOCK.
Strukturovane zamykani - hierarchie/strom.
Casova razitka.

Optimisticky protokol: 1. cteni, 2. kontrola, 3. zapis.

TODO: "metody navrhu IS"?
TODO: prehled SQL
